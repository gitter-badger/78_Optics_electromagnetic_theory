\documentclass[12pt]{article}
\usepackage{pmmeta}
\pmcanonicalname{DerivationOfADefiniteIntegralFormulaUsingTheMethodOfExhaustion}
\pmcreated{2013-03-22 14:56:35}
\pmmodified{2013-03-22 14:56:35}
\pmowner{ruffa}{7723}
\pmmodifier{ruffa}{7723}
\pmtitle{derivation of a definite integral formula using the method of exhaustion}
\pmrecord{22}{36634}
\pmprivacy{1}
\pmauthor{ruffa}{7723}
\pmtype{Derivation}
\pmcomment{trigger rebuild}
\pmclassification{msc}{78A45}
\pmclassification{msc}{30B99}
\pmclassification{msc}{26B15}

% this is the default PlanetMath preamble.  as your knowledge
% of TeX increases, you will probably want to edit this, but
% it should be fine as is for beginners.

% almost certainly you want these
\usepackage{amssymb}
\usepackage{amsmath}
\usepackage{amsfonts}

% used for TeXing text within eps files
%\usepackage{psfrag}
% need this for including graphics (\includegraphics)
%\usepackage{graphicx}
% for neatly defining theorems and propositions
%\usepackage{amsthm}
% making logically defined graphics
%%%\usepackage{xypic}

% there are many more packages, add them here as you need them

% define commands here
\begin{document}
    The area under an arbitrary function $f(x)$ that is piecewise continuous on $[a,b]$ can be "exhausted" with triangles.  The first triangle has vertices at $(a,0)$ and $(b,0)$, and intersects $f(x)$ at 
\[
x = a + \frac{b - a}{2},
\]
 yielding the estimate
	\[
A_1  = \frac{1}{2}(b - a)f(a + \frac{b - a}{2})
\]


    The second approximation involves two triangles, each sharing two vertices with the original triangle, and intersecting $f(x)$ at \[
x = a + \frac{b - a}{4}
\]
and \[
x = a + \frac{3(b - a)}{4},
\]
 adding the area:

	\[
A_2  = \frac{1}{4}(b - a)\{ f(a + \frac{b - a}{4}) - f(a + \frac{b - a}{2}) + f(a + \frac{3(b - a)}{4})\}
\]

    A third such approximation involves four more triangles, adding the area
	\[
\begin{array}{c}
 A_3 {\rm{ }} = \frac{{{\rm{ }}1}}{8}(b - a)\{ f(a + \frac{b - a}{8}) - f(a + \frac{b - a}{4}) \\ 
  + f(a + \frac{3(b - a)}{8}) - f(a + \frac{b - a}{2}) + f(a + \frac{5(b - a)}{8}) \\ 
  - f(a + \frac{3(b - a)}{4}) + f(a + \frac{7(b - a)}{8})\} . \\ 
 \end{array}
\]

This procedure eventually leads to the formula
\[
\int\limits_a^b {f(x)dx = \sum\limits_{n = 1}^\infty  {A_n }  = \left( {b - a} \right)} \sum\limits_{n = 1}^\infty  {\sum\limits_{m = 1}^{2^n  - 1} {\left( { - 1} \right)^{m + 1} } } 2^{ - n} f\left( {a + m(b - a)/2^n } \right)
\]


{\bf References}

\begin{enumerate}
\item
\PMlinkexternal{http://arxiv.org/abs/math.CA/0011078}{http://arxiv.org/abs/math.CA/0011078}.
\item Int. J. Math. Math. Sci. 31, 345-351, 2002.


\end{enumerate}
%%%%%
%%%%%
\end{document}
